\section{Infrastruktur}
Die Infrastruktur setzt sich aus der Hard- und Software zusammen, welche für die Durchführung der Studienarbeit verwendet wird. So kann die Zusammenarbeit im Team erleichtert werden. Zudem kann die Code Qualität besser überprüft und verbessert werden.
\\
\\
Viele Tools sind bereits aus Modulen oder vergangenen Projektarbeiten bekannt, weshalb man sich schnell auf bestimmte Tools einigen konnte. Trotzdem gibt es noch einige Unklarheiten, da einige Tools von anderen abhängen. Erst im späteren Projektverlauf wird sich zeigen, ob diese Tools verwendet werden. Es kann aber auch sein, dass einige Tools, welche später erwähnt werden, gar nicht im Projekt eingesetzt werden.

\subsection{Dokumentation}
Für das Erstellen der Dokumentation wird \LaTeX verwendet. Dieses Tool erforderte zwar etwas Zeit, um sich mit diesem Tool vertraut zu machen. Das Team ist jedoch überzeugt, dass man mit \LaTeX weniger Probleme hat, wenn man bereits eine gute Vorlage hat.
\\
\\
Die einzelnen Dokumentationen werden jeweils lokal und auf Github gespeichert. Da die Dokumentation auf Github gespeichert ist, kann auch zu jeder Zeit nachverfolgt werden, wann etwas geändert wurde.

\subsection{Arbeitspakete Verwaltung}
Für das Verwalten der Arbeitspakete entschied sich das Team Jira zu verwenden. Für das Engineering Projekt wurde ebenfalls Jira verwendet und es machte einen guten Eindruck. So ist es mit Jira sehr einfach, neue Arbeitspakete zu verfassen und die Arbeitszeit abzuschätzen. Zudem ist das Reporting mit Jira sehr gut und man sehr schnell Burndown Charts erstellen.

\subsection{Konstruktion}
Als Versionsverwaltung wird GIT verwendet, da es zum de facto Standart in der Software Entwicklung gehört. Der grosse Vorteil ist, dass man nicht ständig mit dem Server verbunden sein muss. Zudem konnte jedes Mitglied bereits Erfahrungen mit Github sammeln, was den Einstieg erleichtert.
\\
\\
Als Entwicklungsumgebung wird PyCharm von JetBrains verwendet, da es eine sehr gute IDE für Python ist.

\subsection{Testing}
Sämtlicher Code muss getestet werden, weshalb geeignete Test Frameworks im Frontend sowohl als auch im Backend zum Einsatz kommen. Diese Frameworks müssen sowohl Unit Tests wie auch Mocken können. Mit diesen Vorganen können anschliessend de Micro Tests geschrieben werden.
%TODO Frameworks beschreiben 

\subsection{Infrastruktur}
Das Front- und Backend wird auf Servern des INS laufen. So können die Teilnehmer mit ihrem Webbrowser aber nur auf das Frontend zugreifen, wenn sie sich im selben Netzwerk wie die Server befinden.
\\
\\
Für das Projekt selber wird eine 3-Tier-Architektur angestrebt. Als Datenbank wird  \mbox{MySQL} verwendet.
%TODO Docker beschreiben

\newpage