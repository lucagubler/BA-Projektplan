\section{Qualitätsmanagement}
\subsection{Dokumentation}
Für das Erstellen der Dokumentation wird \LaTeX verwendet. Für den Projektplan und die Dokumentation wird ein eigenes Git Repository eingerichtet. Da die Dokumentation auf Github gespeichert ist, kann auch zu jeder Zeit nachverfolgt werden, wann etwas geändert wurde. \\
Während dem gesamten Projektverlauf wird fortlaufend an der Dokumentation gearbeitet. Somit kann einer hohen Arbeitslast gegen Ende des Projektes entgegengewirkt werden.

\subsection{Source Code}
Der Source Code wird ebenfalls über separates Git Repository verwaltet. Bei jedem Commit wird Travis ausgeführt um zu testen, ob der Code noch läuft.

\subsection{Code Reviews}
Um den Source Code der gesamten Applikation zu verwalten, kommt Git zum Einsatz. So kann sichergestellt werden, dass die Qualität des Codes zu jeder Zeit gewährleistet ist.\\
Branches können nur mit einem Pull Request in den Master Branch eingefügt werden. Somit kann ein Vier-Augen-Prinzip umgesetzt werden, bei welchem die Teammitglieder jeden Code nochmals anschauen müssen, bevor der Code in den Master Branch gemerged werden kann.

\newpage