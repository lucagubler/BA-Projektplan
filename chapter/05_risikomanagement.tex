\section{Risikomanagement}

\subsection{Risiken}
\begin{figure}[H]
	\includegraphics[width=\textwidth,height=\textheight,keepaspectratio]{images/risikoanalyse_vorher.png}
	\caption{Ursprüngliche Risikoanalyse}
\end{figure}

\noindent Die detailierte Risikoanalyse kann dem Dokument ''TechnischeRisiken.xlsx'' entnommen werden.\\
Wie man der Grafik entnehmen kann, ist das Risiko R1 - ''Probleme beim Aufsetzen des Prototypen'' das grösste Risiko. Zum einen ist der Prototyp sehr umfangreich und hat mehrere Komponenten wie das Admin Panel oder die Serverseitigen Applikationen, respektive IIS. Bei diesen Komponenten könnte zum einen fehlendes Know-How zu einem Problem werden. Des weiteren ist der Prototyp mittlerweile 4 Jahre alt. Es könnte durchaus sein, dass API Abfragen nicht mehr funktionieren oder Komponenten veraltet sind. Diese müssten zuerst korrigiert werden, bevor der Prototyp überhaupt läuft.

\subsection{Umgang mit Risiken}
Risiken lassen sich in einem grösseren Projekt leider nicht vermeiden. Allfällige Risiken werden jeweils zu Beginn eines Sprints im Team angesprochen. Falls es als sinnvoll erachtet wird, werden auch gleich Massnahmen ergriffen um das Risiko einzudämmen.\\
Für die wahrscheinlichen Risiken wurden Massnahmen definiert, um den auftretetenden Schaden unter Kontrolle zu bringen. Sollten während des Projektes neue Risiken hervortreten, werden diese in die Risikoanalyse aufgenommen und bewertet.

\subsection{Aktualisierte Risikoanalyse}
Nach der Aufnahme der Risikon wurde versucht, die am höchsten gewichteten Risiken zu minimieren. Wie man der Grafik entnehmen kann, konnten alle Risiken ausser dem Risiko R1, verringert werden.
\begin{figure}[H]
	\includegraphics[width=\textwidth,height=\textheight,keepaspectratio]{images/risikoanalyse_nachher.png}
	\caption{Aktualisierte Risikoanalyse}
\end{figure}

Details können dem Dokument ''TechnischeRisiken.xlsx'' entnommen werden.
\newpage