\section{Management Abläufe}

\subsection{Zeitaufwand}
Die Bachelorarbeit begann in der Woche vom  16. September 2019 und dauert insgesamt 17 Wochen. Für das Erreichen der 12 ECTS ist geplant, dass jedes Teammitglied 360 Stunden arbeitet. Daraus resultiert eine durchschnittliche Arbeitszeit von knapp 24 Stunden pro Woche.

\begin{center}
	\begin{tabular}{| m{5cm} | m{7cm}|}
	\hline
	\textbf{Projektstart} & \textbf{18.2.2019} \\
	\hline
	Projektdauer & 15 Wochen \\
	\hline
	Arbeitsstunden pro Person & 16h pro Woche, Total 240h \\
	\hline
	Arbeitsstunden Total & 480h \\
	\hline
	Projektende & 31.05.2019 \\
	\hline
	\end{tabular}
\end{center}

\noindent Für die Studienarbeit stehen total 480 Stunden zur Verfügung. Mit dem definierten Projektumfang wird diese Arbeitszeit voraussichtlich vollständig ausgenutzt. Sollte der Umfang jedoch früher als erwartet abgeschlossen werden können, kann das Projekt um weitere Funktionalitäten erweitert werden.

\subsection{Zeitplanung}
\subsubsection{Phasen / Sprints}
Das gesamte Projekt wird in die vier Phasen Inception, Elaboration, Construction und Transition eingeteilt. Pro Phase gibt es wiederum einzelne Sprints. Die Sprints starten jeden zweiten Donnerstag und beginnen direkt nach dem Meeting mit den Betreuern. Zudem wurden einzelne Meilensteine definiert, welche auf der untenstehenden Tabelle entnommen werden können.


\subsubsection{Meilensteine}
\begin{center}
	\begin{tabular}{| m{3cm} | m{9cm}|}
	\hline
	\textbf{Datum} & \textbf{Meilenstein} \\
	\hline
	27.02.2019 & M1: Projektplan erstellt \\
	\hline
	15.03.2019 & M2: Domainanalyse und Anforderungsspezifikation erstellt \\
	\hline
	28.03.2019 & M3: End of Elaboration \\
	\hline
	25.04.2019 & M4: Zwischenpräsentation \\
	\hline
	23.05.2019 & M5: End of Construction \\
	\hline
	31.05.2019 & M6: Abgabe der Dokumentation \\
	\hline
	\end{tabular}
\end{center}

\subsubsection{Iterationsplanung}
Zu Beginn jedes Sprints setzt sich das Team zusammen, um den nächsten Sprint zu planen. Dabei wird jeweils besprochen, welche Aufgaben des vergangenen Sprints nicht vollständig abgeschlossen werden konnten. Die nicht abgeschlossenen Arbeiten werden mit neu definierten Aufgaben in den neuen Sprint übernommen und jeweils zeitlich abgeschätzt und priorisiert. Da sich im Team nur 2 Mitglieder befinden, wird darauf verzichtet, die Arbeitspakete unter den Teammitgliedern untereinander zuzuweisen. Die gesamte Planung und Verwaltung der Aufgaben wird in Jira erledigt.

\subsection{Besprechungen}
Die Teammitglieder treffen sich mindestens einmal pro Woche zur Besprechung von Fragen, dem klären von Problemen und um das weitere Vorgehen zu planen. \newline
Diese Meetings finden jeweils am Donnerstag nach dem Meeting mit den Betreuern statt.


\newpage