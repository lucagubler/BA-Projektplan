\section{Qualitäts- und Sicherheitsmassnahmen}
Um sicherzustellen, dass die Qualität der Arbeit immer gewährleistet ist, wurden folgende Massnahmen getroffen:

\begin{center}
	\begin{tabular}{| m{3.5cm} | m{3.5cm} | m{3.5cm} |}
	\hline
	\textbf{Massnahmen} & \textbf{Zeitram} & \textbf{Ziel} \\
	\hline
	\textbf{Meeting} & Wöchentlich & Hier wird der aktuelle Stand des Projektes besprochen und neue Entscheidungen werden gefällt \\
	\hline
	\textbf{Besprechung der Meilensteine} & 7 Mal während der gesamten Projektdauer & Dient zur Kontrolle, ob das Team auf dem richtigen Weg ist \\
	\hline
	\textbf{Code Reviews} & Fortlaufend während der gesamten Projektdauer & Dient zur Erhöhung der Qualität des Codes \\
	\hline
	\textbf{Verwendung aktueller Code Versionen} & Fortlaufend während der gesamten Projektdauer & Bei den aktuellen Versionen wurden Bugs behoben, welche nicht mehr auftreten sollten \\
	\hline
	\end{tabular}
\end{center}

\begin{table}
\center
\begin{tabularx}{\textwidth}{p{0.05\textwidth} p{0.35\textwidth} p{0.6\textwidth}}
\toprule
SW & Meilenstein & Beschreibung \\ \midrule
4 & M0: Kickoff & Start des Projektes \\ 
5 & M1: Besprechung Projektplan & Projektplan erstellt und mit Betreuer \newline besprochen \\
7 & M2: End of Elaboration & Use Cases und funktionale sowie nicht \newline funktionale Anforderungen sind erfasst. \newline Mockups Domainanalyse und Konzept für \newline die Architektur sind erstellt. \\
12 & M3: Feature Freeze & Entwicklung der Features ist \newline abgeschlossen, damit man sich auf Bugfixes \newline und Code Qualität konzentrieren kann.\\
15 & M4: Code Freeze & Entwicklung an der Applikation ist \newline abgeschlossen. End of Construction.\\
17 & M5: Projektende & Abgabe der Bachelorarbeit \\ \bottomrule
\end{tabularx}

\end{table}


\newpage