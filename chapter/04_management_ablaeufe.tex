\section{Management Abläufe}

\subsection{Zeitaufwand}
Die Bachelorarbeit begann in der Woche vom  16. September 2019 und dauert insgesamt 17 Wochen. Für das Erreichen der 12 ECTS ist geplant, dass jedes Teammitglied 360 Stunden arbeitet. Daraus resultiert eine durchschnittliche Arbeitszeit von knapp 24 Stunden pro Woche.

\begin{center}
	\begin{tabular}{| m{5cm} | m{7cm}|}
	\hline
	\textbf{Projektstart} & \textbf{16.09.2019} \\
	\hline
	Projektdauer & 17 Wochen \\
	\hline
	Arbeitsstunden pro Person & 24h pro Woche, Total 360h \\
	\hline
	Arbeitsstunden Total & 720h \\
	\hline
	Projektende & 10.01.2020 \\
	\hline
	\end{tabular}
\end{center}

\noindent Für die Bachelorarbeit stehen total 720 Stunden zur Verfügung. Mit dem definierten Projektumfang wird diese Arbeitszeit voraussichtlich vollständig ausgenutzt. Sollte der Umfang jedoch früher als erwartet abgeschlossen werden können, kann das Projekt um weitere Funktionalitäten erweitert werden.

\subsection{Zeitplanung}
\subsubsection{Phasen / Sprints}
Als Projektmanagement Methode wurde SCRUM gewählt. Das gesamte Projekt wird in die vier Phasen Inception, Elaboration, Construction und Transition eingeteilt. Pro Phase gibt es wiederum einzelne Sprints. Die Sprints starten jeden zweiten Donnerstag und beginnen um 13 Uhr. Zudem wurden einzelne Meilensteine definiert, welche auf der untenstehenden Tabelle entnommen werden können.

\subsubsection{Meilensteine}
\begin{center}
	\begin{tabular}{| m{3cm} | m{9cm}|}
	\hline
	\textbf{Datum} & \textbf{Meilenstein} \\
	\hline
	06.10.2019 & M1: Projektplan erstellt \\
	\hline
	13.10.2019 & M2: Prototyp installiert \\
	\hline
	17.10.2019 & M3: Zwischenpräsentation \\
	\hline
	20.10.2019 & M4: Domainanalyse und Anforderungsspezifikation erstellt \\
	\hline
	27.10.2019 & M5: End of Elaboration \\
	\hline
	20.12.2019 & M6: End of Construction \\
	\hline
	10.01.2020 & M7: Abgabe der Dokumentation \\
	\hline
	\end{tabular}
\end{center}

\subsubsection{Iterationsplanung}
Zu Beginn jedes Sprints setzt sich das Team zusammen, um den nächsten Sprint zu planen. Dabei wird jeweils besprochen, welche Aufgaben des vergangenen Sprints nicht vollständig abgeschlossen werden konnten. Die nicht abgeschlossenen Arbeiten werden mit neu definierten Aufgaben in den neuen Sprint übernommen und jeweils zeitlich abgeschätzt und priorisiert. Da sich im Team nur 2 Mitglieder befinden, wird darauf verzichtet, die Arbeitspakete unter den Teammitgliedern untereinander zuzuweisen. Die gesamte Planung und Verwaltung der Aufgaben wird in Jira erledigt.

\subsection{Besprechungen}
Die Teammitglieder arbeiten an mindestens zwei Tagen pro Woche zusammen im Bachelorarbeits Zimmer. So können Fragen schnell geklärt werden und es kann sich gegenseitig geholfen werden. Im Normalfall findet jeden Donnerstag ein Meeting mit dem Betreuer statt, in dem der aktuelle Stand vorgestellt, Probleme besprochen und das weitere Vorgehen besprochen wird.

\subsection{Versionskontrolle}
%TODO GIT beschreiben -> wie ist der Ablauf? -> 4 Augen Prinzip
Um den Source Code der gesamten Applikation zu verwalten, kommt git zum Einsatz. So kann sichergestellt werden, dass die Qualität des Codes zu jeder Zeit gewährleistet ist. Damit die Teammitglieder immer auf dem neusten Stand sind, wurde beschlossen, dass Merge-Requests unumgänglich sind. Mit dieser Massnahme wird erzwungen, dass jede geschriebene Zeile Code, die in den Master-Branch gemerged werden soll, von mindestens einer weiteren Person zur Kenntniss genommen und überdacht wird. 

\subsection{Projektverwaltung}
%TODO JIRA workflow beschreiben
Als Projektverwaltungstool wird Jira verwendet. Man hat sich für dieses Tool entschieden, da es die gewünschte Funktionalität mit sich bringt und trotzdem sehr schlank und übersichtlich ist. Um den Stand der einzelnen Arbeitspakete möglichst genau darzustellen, wurde ein Workflow definiert, welcher jedes Arbeitspaket durchlaufen muss.
%TODO BILD Workflow einfügen
%TODO ich wür review als workflow schritt use neh. wenn du arbeitspaket erfassisch, denn mussi das ned nomal aluege. review isch guet für code. wenn du doku korrigiersch, wottis ned nomal müsse aluege.
Wie im Bild oben ersichtlich ist, muss jedes Arbeitspaket folgende Status durchlaufen: Open, In Progress, Review und Done. Ein Arbeitspaket kann jeweils nur ein Status vor und/oder zurück verschoben werden. Dies soll verhindern, dass ein Paket direkt vom Status Open in den Status Done verschoben wird. Sobald ein Arbeitspaket den Status Done erreicht hat, kann dies nicht mehr rückgängig gemacht werden. 

\subsection{Code Styleguide}
%TODO Code Styleguide beschreiben
%TODO Quelle: https://marketplace.visualstudio.com/items?itemName=ChrisDahlberg.StyleCop
Bei der Vorgänger Arbeit wurde StyleCop eingesetzt. Dabei handelt es sich um ein Tool, welches bestimmte Style und Konsistenz Regeln erzwingt. Dieses Tool kann direkt in Visual Studio verwendet werden, ohne das Änderungen am Code vorgenommen werden müssen.

\subsection{Performance Tests}
%TODO Last Test vor Arbeit und nachher machen und sagen, was besser ist (oder schlechter)
Um sicherzustellen, dass die Performance der Applikation durch Anpassungen nicht verschlechtert wird, wird beim Beginn der Arbeit ein Performance Test durchgeführt. Diese Performance Tests werden regelmässig durchgeführt, so dass allfällige Verschlechterungen schnell bemerkt und behoben werden können. Die Vorgänger haben bereits eine Applikation geschrieben, welche die Performance testet. Diese Applikation wird auch bei dieser Arbeit wieder verwendet. 

\subsection{Continuous Integration / Continuous Development}
Beim CI/CD Ansatz wird der Code regelmässig in den Master Branch gemerged. Bei diesem Schritt werden die Änderungen automatisch validiert und getestet. Somit kann eine ''integration hell'' vermieden werden, da man nie zu weit vom Master Branch abweicht. \\
Um diesen Ansatz umsetzen zu können, kommt Jenkins zum Einsatz.
\newpage