\section{Management Abläufe}

\subsection{Zeitaufwand}
Die Bachelorarbeit begann in der Woche vom  16. September 2019 und dauert insgesamt 17 Wochen. Für das Erreichen der 12 ECTS ist geplant, dass jedes Teammitglied 360 Stunden arbeitet. Daraus resultiert eine durchschnittliche Arbeitszeit von knapp 24 Stunden pro Woche.

\begin{center}
	\begin{tabularx}{\textwidth}{p{0.4\textwidth} p{0.6\textwidth}}
	\toprule
	Projektstart & 13.10.2019 \\
	Projektdauer & 13 Wochen \\
	Arbeitsstunden pro Person & 23h pro Woche, Total 300h \\
	Arbeitsstunden Total & 600h \\
	Projektende & 10.01.2020 \\ \bottomrule
	\end{tabularx}
	\captionof{table}{Übersicht Zeitaufwand}
\end{center}

\noindent Für die Bachelorarbeit stehen total 720 Stunden zur Verfügung. Da jedoch für das erste Thema ca. 60 Arbeitsstunden pro Person aufgewendet wurden, stehen für das neue Projekt noch total 600 Stunden zur Verfügung. Mit dem definierten Projektumfang wird diese Arbeitszeit voraussichtlich vollständig ausgenutzt. Sollte der Umfang jedoch früher als erwartet abgeschlossen werden können, kann das Projekt um weitere Funktionalitäten erweitert werden.

\subsection{Projektmanagement}
Als Projektmanagement Methode wurde SCRUM+ gewählt. Bei dieser Projektmanagement Methode handelt es sich um einen Mix aus SCRUM und Unified Process. Diese Methode wird auch von Daniel Keller im Modul ''Software Engineering'' unterrichtet.

\newpage

\subsubsection{Phasen / Sprints}
Dabei wird das gesamte Projekt in die vier Phasen Inception, Elaboration, Construction und Transition eingeteilt. Pro Phase gibt es wiederum einzelne Sprints. Zudem wurden einzelne Meilensteine definiert, welche auf der untenstehenden Tabelle entnommen werden können.

\begin{center}
	\begin{tabularx}{\textwidth}{p{0.05\textwidth} p{0.35\textwidth} p{0.6\textwidth}}
	\toprule
	SW & Meilenstein & Beschreibung \\ \midrule
	4 & M0: Kickoff & Start des Projektes \\ 
	5 & M1: Abschluss Projektplan & Projektplan erstellt und mit Betreuer \newline besprochen \\
	7 & M2: End of Elaboration & Use Cases und funktionale sowie nicht \newline funktionale Anforderungen sind erfasst. \newline Mockups Domainanalyse und Konzept für \newline die Architektur sind erstellt. \\
	12 & M3: Feature Freeze & Entwicklung der Features ist \newline abgeschlossen, damit man sich auf Bugfixes \newline und Code Qualität konzentrieren kann.\\
	15 & M4: Code Freeze & Entwicklung an der Applikation ist \newline abgeschlossen. End of Construction.\\
	17 & M5: Projektende & Abgabe der Bachelorarbeit \\ \bottomrule
	\end{tabularx}
	\captionof{table}{Übersicht Meilensteine}
\end{center}

\subsubsection{Iterationsplanung}
Zu Beginn jedes Sprints setzt sich das Team zusammen um den nächsten Sprint zu planen. Dabei wird jeweils besprochen, welche Aufgaben des vergangenen Sprints nicht vollständig abgeschlossen werden konnten. Die nicht abgeschlossenen Arbeiten werden mit neu definierten Aufgaben in den neuen Sprint übernommen und jeweils zeitlich abgeschätzt und priorisiert. Da sich im Team nur 2 Mitglieder befinden, wird darauf verzichtet, die Arbeitspakete unter den Teammitgliedern untereinander zuzuweisen. Die gesamte Planung und Verwaltung der Aufgaben wird in Jira erledigt.

\subsection{Abgabe}
Nachfolgend sind die vorgegebenen Abgabetermine aufgelistet.

\begin{center}
	\begin{tabularx}{\textwidth}{p{0.3\textwidth} p{0.7\textwidth}}
	\toprule
	Datum & Beschreibung \\ \midrule
	06.01.2020 & Erfassung Abstract im Online Tool https://abstract.hsr.ch/ \\
	10.01.2020 & Abage des Berichts an den Betreuer und Hochladen aller Dokumente auf archiv-i.hsr.ch \\ \bottomrule
	\end{tabularx}
	\captionof{table}{Übersicht Abgabetermine}
\end{center}

\subsection{Besprechungen}
Die Teammitglieder arbeiten an mindestens zwei Tagen pro Woche zusammen im Bachelorarbeitszimmer. So können Fragen schnell geklärt werden und es kann sich gegenseitig geholfen werden. Im Normalfall findet jeden Donnerstag ein Meeting mit dem Betreuer statt, in dem der aktuelle Stand vorgestellt, Probleme besprochen und das weitere Vorgehen besprochen wird.

\subsection{Projektverwaltung}
%TODO JIRA workflow beschreiben
Als Projektverwaltungstool wird Jira verwendet. Man hat sich für dieses Tool entschieden, da es die gewünschte Funktionalität mit sich bringt und trotzdem sehr schlank und übersichtlich ist. Um den Stand der einzelnen Arbeitspakete möglichst genau darzustellen, wurde ein Workflow definiert, welcher jedes Arbeitspaket durchlaufen muss.
%TODO BILD Workflow einfügen
%TODO ich wür review als workflow schritt use neh. wenn du arbeitspaket erfassisch, denn mussi das ned nomal aluege. review isch guet für code. wenn du doku korrigiersch, wottis ned nomal müsse aluege.
Wie im Bild ersichtlich ist, muss jedes Arbeitspaket folgende Status durchlaufen: Open, In Progress, Review und Done. Ein Arbeitspaket muss jeden Schritt im Workflow durchlaufen, ausser Review. Dieser Schritt ist optinal und muss nicht bei jedem Arbeitspaket durchgeführt werden.

\subsection{Zeiterfassung}
Die Zeiterfassung wird mit Jira verwaltet. Beim Einfügen von Arbeitspaketen in den Sprint wird die Zeit geschätzt, welche für das Arbeitspaket aufgewendet werden muss. Jede Person, welche an diesem Arbeitspaket gearbeitet hat, kann Zeit auf dieses Arbeitspaket buchen. Am Schluss kann so eine Zeitauswertung über die einzelnen Sprints oder das gesamte Projekt gemacht werden.

\newpage