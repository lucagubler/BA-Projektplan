\section{Projektübersicht}
Es soll das bereits bestehende Ticketing System um zusätzliche Funktionalität erweitert werden. Ziel des Ticketing Systems ist es, Menschen an Orten mit limitiertem Zugang zu modernen Hilfsmitteln zu unterstützen. Dabei können zum Beispiel die Menschen mit Experten einer Hilfsorganisation in Kontakt treten, ohne dass diese physisch vor Ort sein müssen. Somit können sehr viele Menschen mit vergleichsweise niedrigem Zeitaufwand angesprochen werden

\subsection{Zweck und Ziel}
Das Ticketing System wurde bereits in einer früheren Bachelorarbeit umgesetzt. Ziel dieser Bachelorarbeit ist es, das bereits existierende System um zusätzliche Funktionalität zu erweitern.
\medskip
\newline
Folgende Ziele wurden für diese Arbeit gesetzt:
\begin{itemize}
	\itemsep0em
	\item Der erfolgreiche Abschluss der Bachelorarbeit und das Erhalten der 12 ECTS.
	\item Die erlernte Theorie der Module ''Software Engineering 1 \& 2'' in die Praxis umsetzen.
	\item Den Umgang mit neuen Technologien kennen lernen.
\end{itemize}

\subsection{Lieferumfang}
Folgende Dokumente werden am Ende der Bachelorarbeit abgeliefert:
\begin{itemize}
	\itemsep0em
	\item Abstract
	\item Aufgabenstellung
	\item Einverständniserklärung
	\item Erklärung zur Urheberschaft
	\item Passwörter
	\item Persönliche Berichte
	\item Protokolle
	\item Source Code	
\end{itemize}

\subsection{Annahmen und Einschränkungen}
Pro Teammitglied wird mit einem Arbeitsaufwand von 24 Stunden pro Woche gerechnet. Falls jedoch Probleme auftreten oder der Arbeitsaufwand falsch eingeschätzt wurde, kann sich der Arbeitsaufwand auf bis zu 40 Stunden pro Woche erhöhen. Falls es jedoch zu Verzögerungen kommt, wird der Arbeitsumfang in Absprache mit den Betreuern dementsprechend angepasst.

\newpage