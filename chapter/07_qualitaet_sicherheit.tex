\section{Qualitäts- und Sicherheitsmassnahmen}
Um sicherzustellen, dass die Qualität der Arbeit immer gewährleistet ist, wurden folgende Massnahmen getroffen:

\begin{center}
	\begin{tabular}{| m{3.5cm} | m{3.5cm} | m{3.5cm} |}
	\hline
	\textbf{Massnahmen} & \textbf{Zeitram} & \textbf{Ziel} \\
	\hline
	\textbf{Meeting} & Wöchentlich & Hier wird der aktuelle Stand des Projektes besprochen und neue Entscheidungen werden gefällt \\
	\hline
	\textbf{Besprechung der Meilensteine} & 6 Mal während der gesamten Projektdauer & Dient zur Kontrolle, ob das Team auf dem richtigen Weg ist \\
	\hline
	\textbf{Code Reviews} & Fortlaufend während der gesamten Projektdauer & Dient zur Erhöhung der Qualität des Codes \\
	\hline
	\textbf{Verwendung aktueller Code Versionen} & Fortlaufend während der gesamten Projektdauer & Bei den aktuellen Versionen wurden Bugs behoben, welche nicht mehr auftreten sollten \\
	\hline
	\end{tabular}
\end{center}

\subsection{Projektmanagement}
Als Projektmanagementsystem wird Jira verwendet. Dieses Tool erlaubt es, einzelne Arbeitspakete zu erstellen. So hat jedes Teammitglied jederzeit den Überblick über den aktuellen Stand des Projektes. Zudem ist ein Zeit-Management-Tool integriert. Somit kann während dem gesamten Projektverlauf der Zeitaufwand im Auge behalten werden.

\subsubsection{Workflows}

\subsection{Entwicklung}

\subsubsection{Vorgehen}

\subsubsection{Unit Testing}

\subsubsection{Code Reviews}

\subsubsection{Code Style Guides}

\newpage