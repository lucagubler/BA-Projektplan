\section{Infrastruktur}
Die Infrastruktur setzt sich aus der Hard- und Software zusammen, welche für die Durchführung der Bachelorarbeit verwendet wird. So kann die Zusammenarbeit im Team erleichtert werden. Zudem kann die Code Qualität besser überprüft und verbessert werden.
\\
Viele Tools sind bereits aus Modulen oder vergangenen Projektarbeiten bekannt, weshalb man sich schnell auf bestimmte Tools einigen konnte. Trotzdem gibt es noch einige Unklarheiten, da einige Tools von anderen abhängig sind. Erst im späteren Projektverlauf wird sich zeigen, ob diese Tools verwendet werden. Es kann aber auch sein, dass einige Tools, welche erwähnt werden, gar nicht im Projekt eingesetzt werden.

\subsection{Arbeitspakete Verwaltung}
Für das Verwalten der Arbeitspakete entschied sich das Team, Jira zu verwenden. Für das Engineering Projekt und die Studienarbeit wurde ebenfalls Jira verwendet. So ist es mit Jira sehr einfach, Workflows zu definieren, neue Arbeitspakete zu erfassen und die Arbeitszeit auszuwerten. Zudem ist das Reporting mit Jira sehr gut und man sehr schnell Burndown Charts erstellen.

\subsection{Konstruktion}
Als Versionsverwaltung wird GIT verwendet, da es zum de facto Standart in der Software Entwicklung gehört. Der grosse Vorteil ist, dass man nicht ständig mit dem Server verbunden sein muss. Zudem konnte jedes Mitglied bereits Erfahrungen mit Github sammeln, was den Einstieg erleichtert.
\\
Als Entwicklungsumgebung wird Visual Studio von Microsoft verwendet.

\subsection{Testing}
Sämtlicher Code muss getestet werden. Da die Team Mitglieder jedoch noch nicht viel Erfahrung in der C\# / ASP.NET Entwicklung haben, muss dieser Punkt jedoch noch genauer evaluiert werden.
%TODO Frameworks beschreiben 

\subsection{Infrastruktur}
Da sich das Projekt aus mehreren Applikationen zusammen setzt, werden diese folglich auch an unterschiedlichen Orten eingesetzt. Das Admin Panel wird auf einem Windows Client ausgeführt und das Mobile App auf einem Android Handy. Die Server Applikation wird auf einem Windows Server deployed und als Datenbank kommt MS SQL Server zum Einsatz.
\\
Das Ziel dieser Arbeit ist es, die bereits existierende Applikation um neue Funktionalitäten zu erweiteren. Schwerwiegende Änderungen an der Software Architektur sollen folglich vermieden werden.

\newpage